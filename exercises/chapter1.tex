\section{The Real and Complex Number Systems}

% Problem 1.1
\begin{problem}
  If $r$ is rational $(r \neq 0)$ and $x$ is irrational, prove that $x + r$ and $rx$ are irrational.
\end{problem}

\begin{solution}
  Suppose, for a contradiction, that $r + x$ was rational.
  Then we could write $r + x = \frac{p}{q}$, for some integers $p,q$.
  Since $r$ is rational we can write it as $\frac{a}{b}$ for some integers $a,b$.
  Now we know that $\frac{a}{b} + x = \frac{p}{q}$.
  Some algebra gives us $x=\frac{pb-ab}{qb}$.
  Since $a, b, p, q$ are integers, and integers are closed under addition and multiplication, we know that $x$ is rational, a contradiction.

  Similarly for multiplication we could show that $x=\frac{pb}{aq}$, which leads to the same contradiction.
\end{solution}

% Problem 1.2
\begin{problem}
  Prove that there is no rational number whose square is 12.
\end{problem}

\begin{solution}
  Let's call the number whose square is $3$ "$\sqrt{3}$".
  If $\sqrt{3}$ is irrational then by exercise 1.1, $2 \cdot \sqrt{3}$ is also irrational, and $(2 \cdot \sqrt{3})^2 = 4 \cdot 3 = 12$.
  Let's assume, for a contradiction, that $\sqrt{3} = \frac{p}{q}$, for some $p,q \in \N$, such that $p,q$ are relatively prime.
  From there we can get $p^2 = 3 \cdot q^2$. This shows that $p^2$ is divisible by $3$, hence $p$ is divisible by 3. Now we can write $p = 3k$ and $p^2 = 9k^2$ for some $k \in \N$.
  Combining $p^2 = 3q^2$ and $p^2 = 9k^2$ we get $q^2 = 3k^2$, which implies that $q = 3r$, for some $r \in \N$.
  This contradicts our assumption that $p$ and $q$ are relatively prime.
\end{solution}

% Problem 1.3
\begin{problem}
  Prove Proposition 1.15.

  The axioms for multiplication imply the following statements.
  \begin{enumerate}[label=(\alph*)]
    \item If $x \neq 0$ and $xy = xz$ then $y = z$.
    \item If $x \neq 0$ and $xy = x$ then $y = 1$.
    \item If $x \neq 0$ and $xy = 1$ then $y = \frac{1}{x}$.
    \item If $x \neq 0$ then $\frac{1}{\frac{1}{x}} = x$.
  \end{enumerate}
\end{problem}

\begin{solution}
  \begin{enumerate}[label=(\alph*)]
    \item \[y = y \cdot 1 = y \cdot (x \cdot \frac{1}{x}) = (y \cdot x) \cdot \frac{1}{x} = (x \cdot y) \cdot \frac{1}{x} = (x \cdot z) \cdot \frac{1}{x} = (z \cdot x) \cdot \frac{1}{x} = z \cdot (x \cdot \frac{1}{x}) = z \cdot 1 = z.\]
    \item Follows directly from 1.15 (a) with $z = 1$.
    \item Follows directly from 1.15 (a) with $z = \frac{1}{x}$.
    \item From M5 $x \cdot \frac{1}{x} = 1$. From 1.15 (c) with $y = x$ we have $x = \frac{1}{\frac{1}{x}}$.
  \end{enumerate}
\end{solution}

% Problem 1.4
\begin{problem}
  Let $E$ be a nonempty subset of an ordered set.
  Suppose $\alpha$ is a lower bound of $E$ and $\beta$ is an upper bound of $E$.
  Prove that $\alpha \le \beta$.
\end{problem}

\begin{solution}
  Since $E$ is nonempty there must be some $x \in E$.
  By the definition of lower bound $\alpha \le x$. Similarly we know $\beta \ge x$.
  By transitivity we know $\alpha \le \beta$.
\end{solution}

% Problem 1.5
\begin{problem}
  Let $A$ be a nonempty set of real numbers which is bounded below.
  Let $-A$ be the set of all numbers $-x$, where $x \in A$.
  Prove that $inf(A) = -sup(-A)$.
\end{problem}

\begin{solution}
  Since $A$ is a nonempty subset of reals that is bounded by below we know that $inf(A)$ exists such that for any $a \in A$, $inf(A) \le a$, and any lower bound $\gamma \le inf(A)$.
  From here we know that $-inf(A) \le -a$, for all $a \in A$.
  This shows us that $-inf(A)$ is an upper bound of $-A$.
  From $\gamma \le inf(A)$ we know that $-\gamma \ge -inf(A)$, and since $\gamma$ is less than $a \in A$ we know that $-\gamma$ is greater any $-a \in -A$, and thus an upper bound of $-A$.
  But since $-\gamma \ge -inf(A)$ for all such $\gamma$, $-inf(A)$ must be the supremum of $-A$.
\end{solution}

% Problem 1.6
\begin{problem}
  Fix $b > 1$.
  \begin{enumerate}[label=(\alph*)]
    \item If $m, n, p, q$ are integers, $n > 0$, $q > 0$, and $r = \frac{m}{n} = \frac{p}{q}$, prove that $(b^m)^{\frac{1}{n}} = (b^p)^{\frac{1}{q}}$ hence it makes sense to define $b^r = (b^m)^{\frac{1}{n}}$.
    \item Prove that $b^{r + s} = b^rb^s$ if $r$ and $s$ are rational.
    \item If $x$ is real, define $B(x)$ to be the set of all numbers $b^t$, where $t$ is rational and $t \le x$. Prove that $b^r = sup(B(r))$ when $r$ is rational. Hence it makes sense to define $b^x = sup(B(x))$ for every real $x$.
    \item Prove that $b^{x + y} = b^xb^y$ for all real $x$ and $y$.
  \end{enumerate}
\end{problem}

\begin{solution}
  \begin{enumerate}[label=(\alph*)]
    \item We know there exists one and only one $y \in \R$ such that $y^n = b^m$, and similarly one $z \in \R$ such that $z^p = b^p$.
      Consider $y^{nq} = (y^n)^q = (b^m)^q = b^{mq} = b^{np} = (b^p)^n = z^{qn}$.
      This implies that $y = z$, hence $(b^m)^{\frac{1}{n}} = (b^p)^{\frac{1}{q}}$.
    \item Since $r, s \in \Q$ we can write $r = \frac{m}{n}$ and $s = \frac{p}{q}$.
      Now we know that there exists a unique $y \in \R$ such that $y^n = b^m$, similarly there is a $z \in \R$ such that $z^q = b^p$.
      We know that \[b^{r + s} = b^{\frac{m}{n} + \frac{p}{q}} = b^{\frac{mq + pn}{nq}} = (b^{mq + pn})^{\frac{1}{nq}} = (b^{mq} \cdot b^{pn})^{\frac{1}{nq}} = b^{\frac{mq}{nq}} \cdot b^{\frac{pn}{nq}} = b^{\frac{m}{n}} \cdot b^{\frac{p}{q}} = b^rb^s.\]
    \item Clearly $B(r)$ is nonempty since $b^r \in B(r)$.
      Since $b > 1$ and $r \ge t$ for every rational $t$ in $B(r)$ it is clear that $b^r$ is an upper bound of $B(r)$.
      Suppose there was an upper bound $\gamma < b^r$.
      However, since $b^r \in B(r)$, we know that $\gamma$ can't be an upper bound. Hence $b^r = sup(B(r))$.
    \item Since $b^x = sup(B(x))$ we know that $b^x \ge b^r$ for all rational $r \le x$.
      Similarly $b^y \ge b^s$, and $b^{x + y}$ is the supremum of $B(x + y)$, or $b^{x + y} \ge b^{r + s}$ for $(r \le x) \in \Q$ and $(s \le y) \in \Q$.
      From those we know $b^x \cdot b^y \ge b^r \cdot b^s = b^{r + s}$ for every $r \le x$ and $s \le y$, hence $b^xb^y$ is an upper bound of $B(x + y)$ and $sup(B(x + y)) \le b^xb^y$ or $b^{x + y} \le b^xb^y$.
      Now consider $\frac{b^{x + y}}{b^r} \ge b^s$.
      We can see that $\frac{b^{x + y}}{b^r}$ is an upper bound of $B(s)$ and hence $\frac{b^{x + y}}{b^r} \ge b^y$.
      Similarly we get $\frac{b^{x + y}}{b^y} \ge b^x$.
      From here it's clear that $b^{x + y} \ge b^xb^y$.
      Thus we have $b^{x + y} = b^xb^y$.
  \end{enumerate}
\end{solution}

% Problem 1.8
\setcounter{problem}{7}
\begin{problem}
  Prove that no order can be defined in the complex field that turns it into an ordered field.
\end{problem}

\begin{solution}
  Assume $i > 0$.
  Then $i \cdot i > i \cdot 0$, which implies $-1 > 0$.
  Then $-1 \cdot -1 > 0 \cdot -1$ or $1 > 0$.
  However, we know that $1 = 1 + 0 < 1 + (-1) = 0 < 1$.
  This contradicts transitivity, hence $i \le 0$.
  Clearly $i \neq 0$, so $i < 0$.
  From there $-i > 0$, but $(-i) \cdot (-i) > 0 \cdot (-i)$ implies $-1 > 0$, which is , again, a contradiction.
  Hence no order can turn $\C$ into an ordered field.
\end{solution}

% Problem 1.9
\begin{problem}
  Suppose $z = a + bi, w = c + di$.
  Define $z < w$ if $a < c$ or $a = c$ but $b < d$.
  Prove that this turns the set of all complex numbers into an ordered set.
  Does this set have the least upper bound property?
\end{problem}

\begin{solution}
  For $\C$ to be an ordered set with order defined as such it needs to satisfy trichotomy and transitivity.

  First, for trichotomy, take some $z, w \in \C$.
  By the trichotomy of $\R$ we know that only one of $a < c, a = c, a > c$ is true.
  In case $a < c$ we know $z < w$.
  Similarly if $a > c$, then $z > w$.
  If $a = c$, then looking at $b, d \in \R$ we again know that only one of $b < d, b = d, b > d$ is true.
  For $b < d$ we have $z < w$, and for $b > d$ we have $z > w$.
  Finally in case that $a = c$ and $b = d$ then we know $z = w$.

  Second, for transitivity, suppose $z = a + bi, w = c + di, u = e + fi$, such that $z < w$ and $ w < u$.
  We will split this into 4 cases and show that $z < u$ in all of them.
  \begin{enumerate}[label=\textbf{Case \arabic*:}]
    \item $a < c$ and $c < e$. By transitivity in $\R$ we know that $a < e$, hence $z < u$.
    \item $a < c$ and $c = e$ but $d < f$. From there it's clear that $a < e$ and $z < u$.
    \item $a = c$ but $b < d$ and $c < e$. Again, it's clear that $a < e$ and $z < u$.
    \item $a = c = e$ but $b < d$ and $d < f$. By transitivity of $\R$ $b < f$, hence $z < u$.
  \end{enumerate}
  
  Hence $\C$ is an ordered set.

  Now to see if it has the least upper bound property.
  Suppose $w = c + di$ was the supremum of $E = {z = a + bi | b \in \R, a < 0}$.
  Clearly $E \subset \C$, nonempty and bounded above.
  Since we know that $w \ge z$ for every $z \in E$, then either $c \ge a$ for all $a < 0$ or $d \ge b$ for all $b \in \R$.
  It is clear that $w = 0 + di$ is an upper bound of $E$, but since $d \in \R$ and $\R$ is unbounded there will always be some $u = 0 + ei < w$ such that $e < d$.
  Clearly $u$ is also an upper bound of $E$.
  For $c > 0$ it is clear that smaller upper bounds exist.
  So $c$ must be less than $0$.
  However since $c < \frac{c}{2} < 0$ we know that some $u = \frac{c}{2} + ei$ is greater than $w$, contradicting that $w$ is an upper bound of $E$.
  Hence $\C$ doesn't have the least upper bound property.
\end{solution}

% Problem 1.12
\setcounter{problem}{11}
\begin{problem}
  If $z_1, \ldots, z_n$ are complex, prove that \[|z_1 + z_2 + \ldots + z_n| \le |z_1| + |z_2| + \ldots + |z_n|.\]
\end{problem}

\begin{solution}
  By squaring the left side we have \[|z_1 + z_2 + \ldots + z_n|^2 = (z_1 + z_2 + \ldots + z_n)(\overline{z_1} + \overline{z_2} + \ldots + \overline{z_n}) = \sum_{i = 1}^{n} \sum_{j = 1}^{n} z_i\overline{z_j}.\]
  Notice that $z_1\overline{z_1} = |z_1|^2$ and that for $i, j \in \Z$, $1 \le i \le j \le n$, $z_i\overline{z_j} + z_j\overline{z_i} = 2Re(z_i\overline{j})$.
  Since we know that $Re(z_i\overline{j}) \le |z_i\overline{j}| = |z_i||z_j|$ then we know that \[|z_1 + z_2 + \ldots + z_n|^2 \ge |z_1|^2 + |z_2|^2 + \ldots + |z_n|^2 + 2(|z_1||z_2| + |z_1||z_3 + \ldots).\]
  Clearly the right side is, by the binomial theorem, equal to $(|z_1| + |z_2| + \ldots + |z_n|)^2$.
  Taking the roots not gives us $|z_1 + z_2 + \ldots + z_n| \le |z_1| + |z_2| + \ldots |z_n|$, as desired.
\end{solution}

% Problem 1.15
\setcounter{problem}{14}
\begin{problem}
  Under what conditions does equality hold in the Schwarz inequality.
\end{problem}

\begin{solution}
  By the proof of the Schwarz inequality, the equality $AB = |C|^2$ clearly implies $\sum_{j = 1}^{n} |Ba_j - Cb_j|^2 = 0$, i.e.,
  \[|Ba_j - Cb_j|^2 = 0;\]
  \[(Ba_j - Cb_j)(\overline{Ba_j} - \overline{Cb_j}) = 0;\]
  \[B^2|a_j|^2 + C^2|b_j|^2 - 2BCRe(a_jb_j) = 0;\]
  \[0 \ge (B|a_j| - C|b_j|)^2;\]
  Hence we have the equality hold if and only if $B|a_j| = C|b_j|$ if and only if $\frac{|a_j|}{|b_j|}$ is a constant for all $1 \le j \le n$.
\end{solution}

% Problem 1.20
\setcounter{problem}{19}
\begin{problem}
  With reference to the Appendix, suppose that property (III) were ommited from the definition of a cut.
  Keep the same definitions of order and addition.
  Show that the resulting ordered set has the least upper bound property, that addition satisfies axioms (A1) to (A4) (with a slightly different zero-element!) but fails (A5) fails.
\end{problem}

\begin{solution}
  Completness and A1-A3 are trivially the same proofs.
  A4 can also be shown easily with $0^* = {p \in \Q | p \le 0}$.
  We will now show that A5 is false.

  Since $0^*$ has the largest member of 0 we know that our $\alpha + (-\alpha)$ must also have the largest member of $0$.
  Pick $\alpha$ such that it has no largest member.
  If $\alpha + (-\alpha)$ had a largest member $x = r + s$, where $r \in \alpha$ and $s \in -\alpha$ then there would be a $p \in \alpha$ such that $p > r$ and $p + s > r + s \in \alpha + (-\alpha)$, a contradiction.
\end{solution}
