\section{Basic Topology}

% Problem 2.2
\setcounter{problem}{1}
\begin{problem}
  A complex number $z$ is said to be \textit{algebreic} if there are integers $\alpha_0, \ldots, \alpha_n$, not all zero such that
  \[\alpha_0z^n + \alpha_1z^{n - 1} + \ldots + \alpha_{n - 1}z + \alpha_n = 0.\]
  Prove that the set of all algebreic numbers is countable.
\end{problem}

\begin{solution}
  Let $P_n$ be the set of all polynomials with integers coefficants with degree less than or equal to $n$.
  Each polynomial is uniquely determined by it's coefficants, so it is clear that $f(\alpha_0z^n + \ldots + \alpha_n) = (\alpha_0, \alpha_1, \ldots, \alpha_n)$ is a bijection.
  By Theorem 2.13 it is clear that the set $P_n$ is countable.
  Let $B_{p(z)}$ be the set of all roots of the polynomial $p(z)$.
  Since a polynomial of degree $n$ has at most $n$ roots, $B_{p(z)}$ is finite.
  Then we know that $S = \bigcup_{p(z) \in P_n} B_{p(z)}$ is at most countable by Theorem 2.12.
  Since all of the integers are algebreic numbers we know that $S$ is infinite and thus countable.
\end{solution}

% Problem 2.3
\begin{problem}
  Prove that there exist real numbers which are not algebreic.
\end{problem}

\begin{solution}
  If all real numbers were algebreic then clearly the set of all real numbers would be a subset of the set of all algebreic numbers.
  Since, by Theorem 2.8, every infinite subset of the set of algebreic numbers would be countable, then the set of all real numbers must be countable.
  This contradicts Theorem 2.14, thus there must exist real numbers that are not algebreic.
\end{solution}

% Problem 2.4
\begin{problem}
  Is the set of all irrational real numbers countable?
\end{problem}

\begin{solution}
  Suppose the set of all irrational numbers was countable.
  Since we know that the set of all rationals is countable, then their union must also be countable; by a weaker Theorem 2.12.
  However, we know that the union of the set of irrational numbers and the set of rational numbers is simply the set of real numbers.
  We also know by Theorem 2.14 that the set of real numbers is uncountable.
  Thus we arrive at a contradiction, and the set of irrational numbers cannot be countable.
\end{solution}

% Problem 2.5
\begin{problem}
  Construct a bounded set of real numbers with exactly three limit points.
\end{problem}

\begin{solution}
  Consider the set $S$ that is the union of the set $A = \{q \in \Q | q = \frac{1}{n}, \in \N\}$, $B = \{q \in \Q | q = \frac{1}{n} + 1, \in \N\}$ and $C = \{q \in \Q | q = \frac{1}{n} + 2, \in \N\}$.
  The set is clearly bounded and contains only 3 limit points: $0, 1, 2$.
\end{solution}

% Problem 2.6
\begin{problem}
  Let $E'$ be the set of all limit points of a set $E$.
  Prove that $E'$ is closed.
  Prove that $E$ and $\overline{E}$ have the same limit points.
  Do $E$ and $E'$ always have the same limit points?
\end{problem}

\begin{solution}
  Suppose that $E'$ is the set of all limit points of some set $E$.
  We wish to show that any limit point of $E'$ is also a limit point of $E$.
  Let $x$ be a limit point of $E'$.
  Then for any neighborhood $Nr(x)$ there is some $p \neq x$ that is in $E'$.
  Consider the neighborhood of $p$ with radius $r - d(x, p)$.
  There must be some $s \in E$ in the neighborhood.
  Then $d(x, s) \le d(x, p) + d(p, s) < d(x, p) + r - d(x, p)$.
  Hence $s \in Nr(x)$ and $x$ is a limit point of $E'$.
  A similar argument works for $E$ and $\overline{E}$ having the same limit points.
  $E$ and $E'$ obviously do not always have the same limit points.
  Consider the set $A = \{q \in \Q | q = \frac{1}{n}, n \in \N\}$.
  This set has a limit point only in zero, thus $A' = {0}$.
  Any finite set has no limit point, by Theorem 2.20.
\end{solution}

% Problem 2.11
\setcounter{problem}{10}
\begin{problem}
  For $x \in \R^1$ and $y \in \R^1$, define
  \[d_1(x, y) = (x - y)^2,\]
  \[d_2(x, y) = \sqrt{|x - y|},\]
  \[d_3(x, y) = |x^2 - y^2|,\]
  \[d_4(x, y) = |x - 2y|,\]
  \[d_5(x, y) = \frac{|x - y|}{1 + |x - y|}.\]
  Determine, for each of these, whether it is a metric or not.
\end{problem}

\begin{solution}
  $d_1$ satisfies 1) and 2) trivially, but fails 3) since $d_1(1, 3) > d_1(1, 2) + d_1(2, 3)$.
  
  $d_2$ satisfies all 3 conditions and is a metric.

  $d_3$ fails 1) as we can have $d_3(x, -x) = 0$.

  $d_4$ fails 1) as we have $d_4(x, x) \neq 0$ and $d_4(2x, x) = 0$.

  $d_5$ satisfies all 3 conditions with some algebra, and is thus a metric.
\end{solution}

% Problem 2.12
\begin{problem}
  Let $K \subset \R^1$ consist of $0$ and the numbers $\frac{1}{n}$, for $n = 1, 2, 3, \ldots$.
  Prove that $K$ is compact directly from the defintion.
\end{problem}

\begin{solution}
  Let $K = \{0, 1, \frac{1}{2}, \frac{1}{3}, \ldots, \}$ and $\{G_{\alpha}\}$ be an open cover of $K$.
  There must be some $\alpha_1$ such that $0 \in G_{\alpha_1}$.
  Since $G_{\alpha_1}$ is open 0 must be an interior point, thus there exists $a, b \in \R$ where $a < 0 < b$. such that $(a, b) \subset G_{\alpha_1}$.
  By Theorem 1.20(a), there exists a positive integer $N$ such that $Nb > 1$, i.e, $b > \frac{1}{N}$.
  Therefore we have $\frac{1}{n} \in (a, b) \subset G_{\alpha_1}$ for all positive integers $n \ge N$.
  We rewrite $K = \{1, \frac{1}{2}, \ldots, \frac{1}{N - 1}\} \cup \{0, \frac{1}{N}, \frac{1}{N + 1}, \ldots\}$.
  We have already shown that $\{0, \frac{1}{N}, \frac{1}{N + 1}, \ldots\} \subset G_{\alpha_1}$.
  In addition, since $\{1, \frac{1}{2}, \ldots, \frac{1}{N - 1}\}$ is finite set, there are finitely many $G_{\alpha_2}, G_{\alpha_3}, \ldots, G_{\alpha_m}$ such that
  \[\{1, \frac{1}{2}, \ldots, \frac{1}{N - 1}\} \subset G_{\alpha_2} \cup G_{\alpha_3} \cup \ldots \cup G_{\alpha_m}.\]
  Hence we have
  \[K \subset \bigcup_{i = 1}^{m} G_{\alpha_i},\]
  i.e., K is compact by Definition 2.32.
\end{solution}

% Problem 2.13
\begin{problem}
  Construct a compact set of real numbers whose limit points form a countable set.
\end{problem}

\begin{solution}
  We want a closed, bounded set with countably many limit points.
  Consider a set $K \subset \R^2$ such that it contains all points $x = (\frac{1}{p}, \frac{1}{q})$ for $n \in \N$.
  Since this set is clearly bounded and it's limit points are $x = (\frac{1}{k}, 0)$ and $x = (0, \frac{1}{k})$.
  It is also closed, hence by Heine-Borel Theorem $K$ is compact.
  The set of all numbers $\frac{1}{k}$ is an infinite subset of $\Q$, hence it is countable.
  The union of two countable sets is countable by Theorem 2.12.
  Thus $K$ is a compact set with countably many limit points, as desired.
\end{solution}

% Problem 2.14
\begin{problem}
  Give an example of an open cover of the segment (0, 1) which has no finite subcover.
\end{problem}

\begin{solution}
  Let $G_{\alpha}$ be an open set around $\frac{1}{\alpha}$ with radius $\frac{1}{\alpha + 1}$.
  The set $\{G_{\alpha}$ is an open cover of $(0, 1)$ because for any $x \in (0, 1)$ we can find a number $n$ such that $\frac{1}{n + 1} < x < \frac{1}{n}$ (By the Archimeaden property of $\R$), and clearly $x \in G_n$.
  If we removed any $G_{\alpha}$ from this set we would be missing the number $\frac{1}{\alpha}$ in our cover.
\end{solution}

% Problem 2.15
\begin{problem}
  Show that Theorem 2.36 and its Corollary become false if the word "compact" is replaced by "closed" or by "bounded".
\end{problem}

\begin{solution}
  If we replaced the word "compact" by "bounded" we have the following.

  Consider the collection of sets $(0, \frac{1}{n})$ for $n = 1, 2, \ldots$.
  These sets are bounded by $0$ and $1$, and for whichever finite collection of sets we choose we can choose the biggest $n$, and any point in that set will be in the intersection of all our sets.
  However, if there was an $x$ in the intersection of all our sets, we could simply find an $n$ such that $\frac{1}{n} < x$ (By the Archimeaden property of $\R$) and thus $x \notin (0, \frac{1}{n})$ and can't be in the intersection.


  If we replaced the word "compact" by "closed".

  Consider all intervals $[0, n]$ for $n \in \N$.
  The argument is analogous.
\end{solution}

% Problem 2.16
\begin{problem}
  Regard $\Q$, the set of all rational numbers, as a metric space, with $d(x, y) = |p - q|$.
  Let $E$ be the set of all $p \in \Q$ such that $2 < p^2 < 3$.
  Show that $E$ is closed and bounded in $\Q$, but that $E$ is not compact.
  Is $E$ open in $\Q$?
\end{problem}

\begin{solution}
  With some simple algebra we arrive at $E = \{p \in \Q | p \in (-\sqrt{3}, -\sqrt{2}) \cup (\sqrt{2}, \sqrt{3})\}$.
  This set is clearly bounded.
  It is also both open and closed.
  It's obvious that all points are interior points and the same goes for its complement.
  Consider the cover $\{G_{\alpha}\} \cup \{G_{\beta}\}$, where $G_{\alpha}$ is defined as in exercise 2.14, except it goes from $x \in (\sqrt{2}, \sqrt{2} + \frac{1}{\alpha})$.
  Similarly, $G_{\beta}$ is the cover defined with $x \in (-\sqrt{3} + \frac{1}{\alpha}, \sqrt{2})$.
  The set is not compact with the same argument as in exercise 2.14.
\end{solution}
