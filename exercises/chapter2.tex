\section{Basic Topology}

% Problem 2.2
\setcounter{problem}{1}
\begin{problem}
  A complex number $z$ is said to be \textit{algebreic} if there are integers $\alpha_0, \ldots, \alpha_n$, not all zero such that
  \[\alpha_0z^n + \alpha_1z^{n - 1} + \ldots + \alpha_{n - 1}z + \alpha_n = 0.\]
  Prove that the set of all algebreic numbers is countable.
\end{problem}

\begin{solution}
  Let $P_n$ be the set of all polynomials with integers coefficants with degree less than or equal to $n$.
  Each polynomial is uniquely determined by it's coefficants, so it is clear that $f(\alpha_0z^n + \ldots + \alpha_n) = (\alpha_0, \alpha_1, \ldots, \alpha_n)$ is a bijection.
  By Theorem 2.13 it is clear that the set $P_n$ is countable.
  Let $B_{p(z)}$ be the set of all roots of the polynomial $p(z)$.
  Since a polynomial of degree $n$ has at most $n$ roots, $B_{p(z)}$ is finite.
  Then we know that $S = \bigcup_{p(z) \in P_n} B_{p(z)}$ is at most countable by Theorem 2.12.
  Since all of the integers are algebreic numbers we know that $S$ is infinite and thus countable.
\end{solution}

% Problem 2.3
\begin{problem}
  Prove that there exist real numbers which are not algebreic.
\end{problem}

\begin{solution}
  If all real numbers were algebreic then clearly the set of all real numbers would be a subset of the set of all algebreic numbers.
  Since, by Theorem 2.8, every infinite subset of the set of algebreic numbers would be countable, then the set of all real numbers must be countable.
  This contradicts Theorem 2.14, thus there must exist real numbers that are not algebreic.
\end{solution}

% Problem 2.4
\begin{problem}
  Is the set of all irrational real numbers countable?
\end{problem}

\begin{solution}
  Suppose the set of all irrational numbers was countable.
  Since we know that the set of all rationals is countable, then their union must also be countable; by a weaker Theorem 2.12.
  However, we know that the union of the set of irrational numbers and the set of rational numbers is simply the set of real numbers.
  We also know by Theorem 2.14 that the set of real numbers is uncountable.
  Thus we arrive at a contradiction, and the set of irrational numbers cannot be countable.
\end{solution}

% Problem 2.5
\begin{problem}
  Construct a bounded set of real numbers with exactly three limit points.
\end{problem}

\begin{solution}
  Consider the set $S$ that is the union of the set $A = \{q \in \Q | q = \frac{1}{n}, \in \N\}$, $B = \{q \in \Q | q = \frac{1}{n} + 1, \in \N\}$ and $C = \{q \in \Q | q = \frac{1}{n} + 2, \in \N\}$.
  The set is clearly bounded and contains only 3 limit points: $0, 1, 2$.
\end{solution}

% Problem 2.6
\begin{problem}
  Let $E'$ be the set of all limit points of a set $E$.
  Prove that $E'$ is closed.
  Prove that $E$ and $\overline{E}$ have the same limit points.
  Do $E$ and $E'$ always have the same limit points?
\end{problem}

\begin{solution}
  Suppose that $E'$ is the set of all limit points of some set $E$.
  We wish to show that any limit point of $E'$ is also a limit point of $E$.
  Let $x$ be a limit point of $E'$.
  Then for any neighborhood $Nr(x)$ there is some $p \neq x$ that is in $E'$.
  Consider the neighborhood of $p$ with radius $r - d(x, p)$.
  There must be some $s \in E$ in the neighborhood.
  Then $d(x, s) \le d(x, p) + d(p, s) < d(x, p) + r - d(x, p)$.
  Hence $s \in Nr(x)$ and $x$ is a limit point of $E'$.
  A similar argument works for $E$ and $\overline{E}$ having the same limit points.
  $E$ and $E'$ obviously do not always have the same limit points.
  Consider the set $A = \{q \in \Q | q = \frac{1}{n}, n \in \N\}$.
  This set has a limit point only in zero, thus $A' = {0}$.
  Any finite set has no limit point, by Theorem 2.20.
\end{solution}
