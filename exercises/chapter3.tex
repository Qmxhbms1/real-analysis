\section{Numerical Sequences and Seris}


% Problem 3.1
\begin{problem}
  Prove that convergence of $\{s_n\}$ implies convergence of $\{|s_n|\}$.
  Is the converse true?
\end{problem}

\begin{solution}
  By exercise 1.13. we have $||s_n| - |s|| < |s_n - s | < \epsilon$.
  The converse is clearly no true, consider $\{s_n\} = (-1)^n$.
\end{solution}

% Problem 3.2
\begin{problem}
  Calculate $\lim_{n\to\infty} (\sqrt{n^2 + n} - n)$.
\end{problem}

\begin{solution}
  \[\lim_{n\to\infty} \sqrt{n^2 + n} - n = \lim_{n\to\infty} \frac{n^2 + n - n^2}{\sqrt{n^2 + n} + n} = \lim_{n\to\infty} \frac{1}{\sqrt{1 + \frac{1}{n}} + 1} = \frac{1}{2}.\]
\end{solution}

% Problem 3.3
\begin{problem}
  If $s_1 = \sqrt{2}$, and
  \[s_{n + 1} = \sqrt{2 + \sqrt{s_n}}  (n = 1, 2, 3, \ldots),\]
  prove that $\{s_n\}$ converges, and that $s_n < 2$ for $n = 1, 2, 3, \ldots$.
\end{problem}

\begin{solution}
  It is clear that our sequence $\{s_n\}$ is monotonically increasing.
  We shall show inductively that this sequence is bounded by $2$, and hence convergent.

  Base case: $2 > \sqrt{2} = s_1$.

  Inductive step: Let $2 > s_n$ and let's consider $\sqrt{2 + s_n}$.
  By the inductive hypothesis we have $2 = \sqrt{2 + 2} > \sqrt{2 + s_n} = s_{n + 1}$.
  Hence $\{s_n\}$ is convergent, and $s_n < 2$ for $n = 1, 2, 3, \ldots$.
\end{solution}

% Problem 3.4
\begin{problem}
  Find the upper and lower limits of the sequence $\{s_n\}$ defined by
  \[s_1 = 0;\quad s_{2m} = \frac{s_{2m - 1}}{2};\quad s_{2m + 1} = \frac{1}{2} + s_{2m}.\]
\end{problem}

\begin{solution}
  The subsequence $\{s_{2n + 1}\}$ has the recursion relation $s_{m + 1} = \frac{1 + s_m}{2}$, where $m = 2n$ for any $n \in \N$.
  From here we can solve the recurrence relation and get $s_{m} = 1 - \frac{1}{2^{m - 1}}$.
  It is clear to see that this sequence is monotonically increasing and converges to $1$.
  Since it monotonically increases it is clear for every odd $n$ we have $s_{2n + 1} > s_n$.
  If $n$ is even then it is still fairly clear that $s_{2n + 1} > s_{n + 1} > s_{n}$.
  Thus for ever $n$ the subsequence $\{s_{2n + 1}\}$ is greater than or equal to $\{s_n\}$.
  By theorem 3.19 this implies that the $\lim\sup_{n \to \infty} s_n \le \lim\sup_{n \to \infty} s_{2n + 1}$.
  However since $s_{2n + 1}$ is a subsequence of $s_n$ it is clear that it's upper limit $\lim\sup_{n \to \infty} s_{2n + 1} \le \lim\sup_{n \to \infty} s_{n}$.
  Thus $\lim\sup_{n \to \infty} s_{2n + 1} = \lim\sup_{n \to \infty} s_{n}$.
  Since $s_{2n + 1}$ converges to $1$ it is clear that $\lim\sup_{n \to \infty} s_{2n + 1} = 1$, and $\lim\sup_{n \to \infty} s_{n} = 1$.

  Using a similar argument we can show that $\lim\inf_{n \to \infty} s_{n} = \lim\inf_{n \to \infty} s_{2n} = \frac{1}{2}$.
\end{solution}

%Problem 3.5
\begin{problem}
  For any two real sequences $\{a_n\}$, $\{b_n\}$, prove that
  \[\lim\sup_{n \to \infty} (a_n + b_n) \le \lim\sup_{n \to \infty} a_n + \lim\sup_{n \to \infty} b_n,\]
  Provided the sum on the right is not of the form $\infty - \infty$.
\end{problem}

\begin{solution}
  In the case of $sup E_a, sup E_b \in \R$, we use the more general result that states
  \[sup(A + B) \le sup A + sub B.\]
  If one or both are positive infinity, with the other either being positive infinite or a real number, then both sides are equal to positive infinity.
  Similarly if one or both are negative infinity.
  Since we have excluded the case of $\infty - \infty$, these are all the posibilites and in all of the the stated theorem holds.
\end{solution}

%Problem 3.6
\begin{problem}
  Investigate the behavior (convergence or divergence) of $\sum a_n$ if
  \begin{enumerate}[label=(\alph*)]
    \item $a_n = \sqrt{n + 1} - \sqrt{n}$;
    \item $a_n = \frac{\sqrt{n + 1} - \sqrt{n}}{n}$;
    \item $a_n = (\sqrt[n]{n} - 1)^n$;
  \end{enumerate}
\end{problem}

\begin{solution}
  \begin{enumerate}[label=(\alph*)]
    \item If we consider a general partial sum we see that we have a telescoping sum.
      Thus we get $\sum_{n = 1}^m a_n = \sqrt{m + 1} - 1$.
      Since the square root function diverges, so do these partial sums.
      Hence the $\lim \sum_{n = 1}^{m} a_n = + \infty$.
    \item Consider the following
      \[\frac{\sqrt{n + 1} - \sqrt{n}}{n} = \frac{1}{n (\sqrt{n + 1} + \sqrt{n})} \le \frac{1}{2 \sqrt[3]{n}} \le \frac{1}{2 n}.\]
      Since we know that the series of the reciprocals of even numbers converges, by theorem 3.25, so does our series.
    \item By induction we can show that $\frac{3}{2}^n > n$.
      From here it follows that $\sqrt[n]{n} - 1 < \frac{1}{2}$.
      Thus we see that each element of our series is less than the corresponing power of 2.
      Since the geometric series of powers of 2 converges, so does our series.
  \end{enumerate}
\end{solution}

% Problem 3.16
\setcounter{problem}{15}
\begin{problem}
  Fix a positive number $\alpha$.
  Choose $x_1 > \sqrt{\alpha}$, and define $x_2, x_3, x_4, \ldots$, by the recursion formula
  \[x_{n + 1} = \frac{1}{2} (x_n + \frac{\alpha}{x_n}).\]
  \begin{enumerate}[label=(\alph*)]
    \item Prove that $\{x_n\}$ decreases monotonically and that $\lim x_n = \sqrt{\alpha}$.
    \item Put $\epsilon_n = x_n - \sqrt{\alpha}$, and show that
      \[\epsilon_{n + 1} = \frac{\epsilon_n^2}{2x_n} < \frac{\epsilon_n^2}{2\sqrt{\alpha}}\]
      so that, setting $\beta = 2\sqrt{\alpha}$,
      \[\epsilon_n < \beta (\frac{\epsilon_1}{\beta})^{2n}  (n = 1, 2, 3, \ldots).\]
    \item This is a good algorithm for computing square roots, since the recursion formula is simple and the convergence is extremely rapid.
      For example, if $\alpha = 3$ and $x_1 = 2$, show that $\frac{\epsilon_1}{\beta} < \frac{1}{10}$ and that therefore
      \[\epsilon_5 < 4 \cdot 10^{-16},  \epsilon_6 < 4 \cdot 10^{-32}.\]
  \end{enumerate}
\end{problem}

\begin{solution}
  \begin{enumerate}[label=(\alph*)]
    \item Consider
      \[x_{n + 1} - x_n = \frac{1}{2}(x_n + \frac{\alpha}{x_n}) - x_n = \frac{x_n^2 + \alpha}{2x_n} - x_n = \frac{\alpha - x_n^2}{2x_n}\]
      We can inductively show that $x_n > \sqrt{\alpha}$ for all $n \in \N$.
      Hence our above equation is always negative and the sequence is monotonically decreasing.
      Since we know that it is bounded from below by $\sqrt{\alpha}$, by the Monotone convergence theorem we know that $\{x_n\}$ converges.
      Let $\lim_{n\to\infty} x_n = x$.
      Consider our relation
      \[x_{n + 1} = \frac{1}{2} (x_n + \frac{\alpha}{x_n}).\]
      We can take the limit of both sides to get
      \[\lim_{n\to\infty} x_{n + 1} = \lim_{n\to\infty} \frac{1}{2} (x_n + \frac{\alpha}{x_n}).\]
      Since $\lim x_n = \lim x_{n + 1}$, and using some properties of limits we get
      \[x = \frac{x}{2} + \frac{\alpha}{x}.\]
      Which with some simple algebra gets us
      \[x = \sqrt{\alpha}.\]
    \item We have the following
      \[\epsilon_{n + 1} = x_{n + 1} - \sqrt{\alpha} = \frac{1}{2}(x_n + \frac{\alpha}{x_n}) - \sqrt{\alpha} = \frac{x_n^2 + \alpha}{2x_n} - \sqrt{\alpha} = \frac{x_n^2 + \alpha - 2x_n\sqrt{\alpha}} = \frac{(x_n - \sqrt{\alpha})^2}{2x_n} = \frac{\epsilon_n^2}{2x_n} < \frac{\epsilon_n^2}{2\sqrt{\alpha}}.\]
    \item Simply put $\beta = 2\sqrt{3}$ and $\epsilon_1 = 2 - \sqrt{3}$.
      It is clear that $\frac{2 - \sqrt{3}}{2\sqrt{3}} < \frac{1}{10}$.
      Hence $\epsilon_5 < 2\sqrt{3} \cdot 10^{-16} < 4 \cdot 10^{-16}.$
      Similarly we get $\epsilon_6 < 4 \cdot 10^{-32}.$
  \end{enumerate}
\end{solution}

% Problem 3.17
\begin{problem}
  Fix $\alpha > 1$.
  Take $x_1 > \sqrt{\alpha}$, and define
  \[x_{n + 1} = \frac{\alpha = x_n}{1 + x_n} = x_n + \frac{\alpha - x_n^2}{1 + x_n}.\]
  \begin{enumerate}[label=(\alph*)]
    \item Prove that $x_1 > x_3 > x_5 > \ldots$.
    \item Prove that $x_2 < x_4 < x_6 < \ldots$.
    \item Prove that $\lim x_n = \sqrt{\alpha}$.
    \item Compare the rapidity of convergence of this process with the one described in Exercise 16.
  \end{enumerate}
\end{problem}

\begin{solution}
  \begin{enumerate}[label=(\alph*)]
    \item Consider $x_{2n - 1} - x_{2n + 1}$.
      Using the recursion formula twice and some algebra we arrive at the following
      \[x_{2n - 1} - x_{2n + 1} = \frac{2(x_{2n - 1}^2 - \alpha)}{1 + 2x_{2n - 1} + \alpha}.\]
      We can easily see by induction that $x_{2n - 1} > \sqrt{\alpha}$ for all $n \in \N$.
      Combining this with our previous equation we can clearly see that
      \[x_{2n - 1} - x_{2n + 1} > 0,\]
      hence $x_1 > x_3 > x_5 > \ldots$.
    \item The same equation holds in this case except with $x_{2n}$ and $x_{2n - 2}$.
      We can again inductively show that $x_2n < \sqrt{\alpha}$ for all $n \in \N$.
      Thus we get $x_{2n} > x_{2n - 2}$.
    \item Consider the relation derived from our relations in a) and b), $x_{2n} = x_{2n + 2} + \frac{(1 + \alpha)x_{2n} + 2\alpha}{1 + 2x_{2n} + \alpha}$.
      It is clear that everything on the right side converges so we can take the limits of both sides and apply the properties of limits to get
      \[x = \frac{(1 + \alpha)x + 2\alpha}{1 + 2x + \alpha}\].
      Some simple algebra finally leads us to $x = \sqrt{\alpha}$, hence $\lim x_{2n} = \sqrt{\alpha}$
      We can completely analogously show that $\lim x_{2n - 1} = \sqrt{\alpha}$.
      Notice that $\lim x_{2n} - x_{2n - 1} = 0$, hence the sequence is Cauchy and thus convergent.
      Let $\lim x_n = x$.
      We know that every subsequence of $\{x_n\}$ converges to $x$ as well.
      Hence $\{x_{2n}\}$ must also converge to $x$.
      By the uniqueness of limit points we have $\lim x_{2n} = \sqrt{\alpha} = x$ and thus $x = \sqrt{\alpha}$.
    \item Let $\epsilon_n = x_n - \sqrt{\alpha}$.
      Some simple algebra shows us that $\epsilon_{2n} \sim (\frac{1 - sqrt{\alpha}}{1 + \sqrt{\alpha}})^2 \cdot \epsilon_{2n - 2}$.
      Similarly we get $\epsilon_{2n - 1} \sim (\frac{1 - sqrt{\alpha}}{1 + \sqrt{\alpha}})^2 \cdot \epsilon_{2n - 3}$.
      These converge linearly, while the sequence in exercise 16. converged quadratically.
  \end{enumerate}
\end{solution}

% Problem 3.20
\setcounter{problem}{19}
\begin{problem}
  Suppose $\{p_n\}$ is a Cauchy sequence in a metric space $X$, and some subsequence $\{p_{n_i}\}$ converges to a point $p \in X$.
  Prove that the full sequence $\{p_n\}$ converges to $p$.
\end{problem}

\begin{solution}
  Let $\{p_n\}$ be a Cauchy sequence in a metric space $X$ and let $\{p_{n_i}\}$ be a subsequence that converges to some $p \in X$.
  Pick an $\epsilon$-neighborhood around $p$.
  For a contradiction, assume $\{p_n\}$ does not converge to $p$, i.e., for any neighborhood of $p$ there are infinitely many points of $\{p_n\}$ outside of the neighborhood.
  Now pick a point of $\{p_{n}\}$ in our neighborhood such that $n > N$ (Where $N$ is the integer such that $n > N$ and $m > N$ imply $d(p_n, p_m) < \epsilon$).
  This is possible since most points of $\{p_{n_i}\}$ are in this neighborhood.
  Call this point $p_n$ and consider the neighborhood around $p_n$ with radius $\epsilon - d(p, p_n)$.
  Note that this is a subset of our $\epsilon$-neighborhood.
  Since $\{p_n\}$ is a Cauchy sequence, there are only finitely many points outside this $\epsilon$-neighborhood, a contradiciton.
\end{solution}

% Problem 3.21
\begin{problem}
  Prove the following analogue of Theorem 3.10(b): If $\{E_n\}$ is a sequence of closed nonempty and bounded sets in a complete metric splace $X$, if $E_{n+1} \subset E_n$, and if 
  \[\lim_{n \to \infty} \text{diam} E_n = 0.\]
  then $\bigcap_1^{\infty} E_n$ consists of exactly one point.
\end{problem}

\begin{solution}
  We just need to show that $\bigcap E_n$ is nonempty, the rest is identical as in Theorem 3.10(b).

  First, we remove all duplicates, so that $E_{n + 1}$ is a proper subset of $E_n$.
  Note that this does not change the intersection in any way, nor does it change that $\lim_{n \to \infty} \text{diam} E_n = 0$.
  Now, for every $n$ pick a $p_n \in E_n$, such that $p_n \notin E_{n + 1}$ and $p_n \notin E_{n - 1}$.
  Let $G_n \subset E_n$ such that $G_n = \{p_n, p_{n + 1}, p_{n + 2}, \ldots\}$.
  Since $G_n \subset E_n$, clearly $\text{diam} G_n \le \text{diam} E_n$, hence $\lim \text{diam} G_n = 0$.
  Thus, $\{p_n\}$ is a Cauchy sequence, and since $X$ is a complete metric space, $\{p_n\}$ converges to some point $p$.
  Since every $E_n$ is closed, if $p \notin E_i$ for some $i \in \N$, then $p$ is not a limit point of $E_i$.
  However, since the sequence of points $\{p_n\}$ starting from $i$ converges to $p$, clearly $p$ is a limit point of $E_i$, a contradiction.
  Thus for every $n \in N$, $p \in E_n$, and finally $p \in \bigcup E_n$.
\end{solution}

% Problem 3.22
\begin{problem}
  Suppose $X$ is a nonempty complete metric space, and $\{G_n\}$ is a sequence of dense open subsets of $X$.
  Prove Baire's theorem, namely, that $\bigcup_1^{\infty} G_n$ is not empty.
\end{problem}

\begin{solution}
  Since $G_1$ is dense it must be nonempty, pick some point $x_1$.
  Since $G_1$ is open, there must be some neighborhood of $x_1$ contained strictly in $G_1$.
  Let this be the neighborhood with radius $r_1$, i.e., the set $\{p \in G_1 | d(x_1, p) < r_1\}$.
  Clearly the closed ball $\overline{B_1} = \{p \in G_1 | d(x_1, p) \le \frac{r_1}{2}\}$ is a subset of our open ball.

  For a open and closed balls $B_n = \{p \in G_n | d(x_n, p) < r_n\}$, $\overline{B_n} = \{p \in G_n | d(x_n, p) \le \frac{r_n}{2}\}$, the intersection of $G_{n + 1} \cap B_n$ must be nonempty.
  Pick an element of the intersection $x_{n + 1}$.
  Since the intersection of two open sets is open, there must be an open ball around $x_{n + 1}$.
  Let $B_{n + 1} = \{p \in G_{n + 1} | d(x_{n + 1}, p) < r_{n + 1}\}$, with $r_{n + 1} \le \frac{x_n}{2}$.
  Construct $\overline{B_{n + 1}} = \{p \in G_{n + 1} | d(x_{n + 1}, p) \le \frac{r_{n + 1}}{2}\}$.
  We can clearly see that $\overline{B_{n + 1}} \subset \overline{B_n}$.

  Thus, this process leaves us with a sequence of closed, bounded sets $\{\overline{B_n}\}$ in a complete metric space.
  Note that the $\lim r_n = 0$ since we forced $r_{n + 1} < \frac{r}{2}$.
  As we have shown in exercise 3.21, the intersection of this sequence in nonempty.
  Notice that each $\overline{B_n} \subset G_n$, by construction.
  Hence $\bigcup \overline{B_n} \subset \bigcup G_n$.
  Finally, we can conclude that $\bigcup G_n$ is nonempty.
\end{solution}

% Problem 3.23
\begin{problem}
  Suppose $\{p_n\}$ and $\{q_n\}$ are Cauchy sequences in a metric space $X$.
  Show that the sequence $\{d(p_n, q_n)\}$ converges.
\end{problem}

\begin{solution}
  Let $\epsilon > 0$ be given.
  Since $\{p_n\}$ and $\{q_n\}$ are Cauchy, there exist $N_1, N_2 \in \N$ such that $m, n > N_1$ implies $d(p_n, p_m) < \frac{\epsilon}{2}$, and similarly $m, n > N_2$ implies $d(q_n, q_m) < \frac{\epsilon}{2}$.
  Put $N = max(N_1, N_2)$.
  Now we have $m, n > N$ implies both $d(p_n, p_m) < \frac{\epsilon}{2}$ and $d(q_n, q_m) < \frac{\epsilon}{2}$.
  By the triangle inequality we have the following:
  \[d(p_n, q_n) \le d(p_n, p_m) + d(p_m, q_m) + d(q_m, q_n).\]
  Which gives us
  \[d(p_n, q_n) - d_(p_m, q_m) \le d(p_n, p_m) + d(q_n, q_m).\]
  Completely analogously we can get
  \[d(p_m, q_m) - d(p_n, q_n) \le d(p_n, p_m) + d(q_n, q_m).\]
  Hence
  \[|d(p_n, q_n) - d(p_m, q_m)| \le d(p_n, p_m) + d(q_n, q_m) < \frac{\epsilon}{2} + \frac{\epsilon}{2} = \epsilon.\]
  Thus the sequence $\{d(p_n, q_n)\}$ is Cauchy, and since $\R$ with the standard metric is a complete metric space, $\{d(p_n, q_n)\}$ converges.
\end{solution}

