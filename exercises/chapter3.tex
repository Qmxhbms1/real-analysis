\section{Numerical Sequences and Seris}


% Problem 3.1
\begin{problem}
  Prove that convergence of $\{s_n\}$ implies convergence of $\{|s_n|\}$.
  Is the converse true?
\end{problem}

\begin{solution}
  By exercise 1.13. we have $||s_n| - |s|| < |s_n - s | < \epsilon$.
  The converse is clearly no true, consider $\{s_n\} = (-1)^n$.
\end{solution}

% Problem 3.2
\begin{problem}
  Calculate $\lim_{n\to\infty} (\sqrt{n^2 + n} - n)$.
\end{problem}

\begin{solution}
  \[\lim_{n\to\infty} \sqrt{n^2 + n} - n = \lim_{n\to\infty} \frac{n^2 + n - n^2}{\sqrt{n^2 + n} + n} = \lim_{n\to\infty} \frac{1}{\sqrt{1 + \frac{1}{n}} + 1} = \frac{1}{2}.\]
\end{solution}

% Problem 3.3
\begin{problem}
  If $s_1 = \sqrt{2}$, and
  \[s_{n + 1} = \sqrt{2 + \sqrt{s_n}}  (n = 1, 2, 3, \ldots),\]
  prove that $\{s_n\}$ converges, and that $s_n < 2$ for $n = 1, 2, 3, \ldots$.
\end{problem}

\begin{solution}
  It is clear that our sequence $\{s_n\}$ is monotonically increasing.
  We shall show inductively that this sequence is bounded by $2$, and hence convergent.

  Base case: $2 > \sqrt{2} = s_1$.

  Inductive step: Let $2 > s_n$ and let's consider $\sqrt{2 + s_n}$.
  By the inductive hypothesis we have $2 = \sqrt{2 + 2} > \sqrt{2 + s_n} = s_{n + 1}$.
  Hence $\{s_n\}$ is convergent, and $s_n < 2$ for $n = 1, 2, 3, \ldots$.
\end{solution}

% Problem 3.16
\setcounter{problem}{15}
\begin{problem}
  Fix a positive number $\alpha$.
  Choose $x_1 > \sqrt{\alpha}$, and define $x_2, x_3, x_4, \ldots$, by the recursion formula
  \[x_{n + 1} = \frac{1}{2} (x_n + \frac{\alpha}{x_n}).\]
  \begin{enumerate}[label=(\alph*)]
    \item Prove that $\{x_n\}$ decreases monotonically and that $\lim x_n = \sqrt{\alpha}$.
    \item Put $\epsilon_n = x_n - \sqrt{\alpha}$, and show that
      \[\epsilon_{n + 1} = \frac{\epsilon_n^2}{2x_n} < \frac{\epsilon_n^2}{2\sqrt{\alpha}}\]
      so that, setting $\beta = 2\sqrt{\alpha}$,
      \[\epsilon_n < \beta (\frac{\epsilon_1}{\beta})^{2n}  (n = 1, 2, 3, \ldots).\]
    \item This is a good algorithm for computing square roots, since the recursion formula is simple and the convergence is extremely rapid.
      For example, if $\alpha = 3$ and $x_1 = 2$, show that $\frac{\epsilon_1}{\beta} < \frac{1}{10}$ and that therefore
      \[\epsilon_5 < 4 \cdot 10^{-16},  \epsilon_6 < 4 \cdot 10^{-32}.\]
  \end{enumerate}
\end{problem}

